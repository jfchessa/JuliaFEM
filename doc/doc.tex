% !TEX TS-program = pdflatex
% !TEX encoding = UTF-8 Unicode

% This is a simple template for a LaTeX document using the "article" class.
% See "book", "report", "letter" for other types of document.

\documentclass[11pt]{article} % use larger type; default would be 10pt

\usepackage[utf8]{inputenc} % set input encoding (not needed with XeLaTeX)

%%% Examples of Article customizations
% These packages are optional, depending whether you want the features they provide.
% See the LaTeX Companion or other references for full information.

%%% PAGE DIMENSIONS
\usepackage{geometry} % to change the page dimensions
\geometry{a4paper} % or letterpaper (US) or a5paper or....
% \geometry{margin=2in} % for example, change the margins to 2 inches all round
% \geometry{landscape} % set up the page for landscape
%   read geometry.pdf for detailed page layout information

\usepackage{graphicx} % support the \includegraphics command and options

% \usepackage[parfill]{parskip} % Activate to begin paragraphs with an empty line rather than an indent

%%% PACKAGES
\usepackage{booktabs} % for much better looking tables
\usepackage{array} % for better arrays (eg matrices) in maths
\usepackage{paralist} % very flexible & customisable lists (eg. enumerate/itemize, etc.)
\usepackage{verbatim} % adds environment for commenting out blocks of text & for better verbatim
\usepackage{subfig} % make it possible to include more than one captioned figure/table in a single float
% These packages are all incorporated in the memoir class to one degree or another...

%%% HEADERS & FOOTERS
\usepackage{fancyhdr} % This should be set AFTER setting up the page geometry
\pagestyle{fancy} % options: empty , plain , fancy
\renewcommand{\headrulewidth}{0pt} % customise the layout...
\lhead{}\chead{}\rhead{}
\lfoot{}\cfoot{\thepage}\rfoot{}

%%% SECTION TITLE APPEARANCE
\usepackage{sectsty}
\allsectionsfont{\sffamily\mdseries\upshape} % (See the fntguide.pdf for font help)
% (This matches ConTeXt defaults)

%%% ToC (table of contents) APPEARANCE
\usepackage[nottoc,notlof,notlot]{tocbibind} % Put the bibliography in the ToC
\usepackage[titles,subfigure]{tocloft} % Alter the style of the Table of Contents
\renewcommand{\cftsecfont}{\rmfamily\mdseries\upshape}
\renewcommand{\cftsecpagefont}{\rmfamily\mdseries\upshape} % No bold!

\setlength{\parindent}{0cm}
\setlength{\parskip}{1em}
%%% END Article customizations

%%% The "real" document content comes below...

\title{Documentation for Julia Finite Element Code}
\author{Jack Chessa}
%\date{} % Activate to display a given date or no date (if empty),
         % otherwise the current date is printed 

\begin{document}
\maketitle

\section{Overview}


\texttt{REALTYPE},  \texttt{IDTYPE}

\section{Basic Finite Element Routines}
\subsection{Shape Function Routines}
Routines that generate element shape functions and gradient with respect to the element coordiante system
\begin{itemize}
\item \texttt{shape\_}\textit{elemtype}\texttt{!(N, xi)}
\item \texttt{dshape\_}\textit{elemtype}\texttt{!(dNdxi, xi)}
\end{itemize}
Element types (\textit{elemtype}) are \texttt{line2}, \texttt{line3}, \texttt{tria3}, \texttt{tria6}, \texttt{quad4}, \texttt{tetra4},   \texttt{tetra10}, \texttt{hexa8}, \texttt{hexa20}.

\subsection{Quadrature functions}

\subsection{Finite element operations}
\begin{itemize}
\item \texttt{setsctr!(sctr, conn, nn, ndofpn)}
\item \texttt{fesolve!(K, f, ifix)}
\item \texttt{fesolve!{T}(K, f::Array{T,1}, ifix::Array{Int,1}, ival::Array{T,1})}
\end{itemize}

\texttt{DelayedAssmMat} functions
\begin{itemize}
\item \texttt{addlocalmat!(K, ke, sctr)}
\item \texttt{addlocalmat!(K::DelayedAssmMat, ke, sctr)}
\item \texttt{getCSCMat(K::DelayedAssmMat)}
\end{itemize}


\begin{itemize}
\item \texttt{addrhs!(loads::Array{LoadPt}, f, dofmap=FixedDOFMap(6))}
\end{itemize}

\end{document}
