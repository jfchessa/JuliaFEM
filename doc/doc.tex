% !TEX TS-program = pdflatex
% !TEX encoding = UTF-8 Unicode

% This is a simple template for a LaTeX document using the "article" class.
% See "book", "report", "letter" for other types of document.

\documentclass[11pt]{article} % use larger type; default would be 10pt

\usepackage[utf8]{inputenc} % set input encoding (not needed with XeLaTeX)

%%% Examples of Article customizations
% These packages are optional, depending whether you want the features they provide.
% See the LaTeX Companion or other references for full information.

%%% PAGE DIMENSIONS
\usepackage{geometry} % to change the page dimensions
\geometry{a4paper} % or letterpaper (US) or a5paper or....
% \geometry{margin=2in} % for example, change the margins to 2 inches all round
% \geometry{landscape} % set up the page for landscape
%   read geometry.pdf for detailed page layout information

\usepackage{graphicx} % support the \includegraphics command and options

% \usepackage[parfill]{parskip} % Activate to begin paragraphs with an empty line rather than an indent

%%% PACKAGES
\usepackage{booktabs} % for much better looking tables
\usepackage{array} % for better arrays (eg matrices) in maths
\usepackage{paralist} % very flexible & customisable lists (eg. enumerate/itemize, etc.)
\usepackage{verbatim} % adds environment for commenting out blocks of text & for better verbatim
\usepackage{subfig} % make it possible to include more than one captioned figure/table in a single float
% These packages are all incorporated in the memoir class to one degree or another...

%%% HEADERS & FOOTERS
\usepackage{fancyhdr} % This should be set AFTER setting up the page geometry
\pagestyle{fancy} % options: empty , plain , fancy
\renewcommand{\headrulewidth}{0pt} % customise the layout...
\lhead{}\chead{}\rhead{}
\lfoot{}\cfoot{\thepage}\rfoot{}

%%% SECTION TITLE APPEARANCE
\usepackage{sectsty}
\allsectionsfont{\sffamily\mdseries\upshape} % (See the fntguide.pdf for font help)
% (This matches ConTeXt defaults)

%%% ToC (table of contents) APPEARANCE
\usepackage[nottoc,notlof,notlot]{tocbibind} % Put the bibliography in the ToC
\usepackage[titles,subfigure]{tocloft} % Alter the style of the Table of Contents
\renewcommand{\cftsecfont}{\rmfamily\mdseries\upshape}
\renewcommand{\cftsecpagefont}{\rmfamily\mdseries\upshape} % No bold!

\setlength{\parindent}{0cm}
\setlength{\parskip}{1em}
%%% END Article customizations

%%% The "real" document content comes below...

\title{Documentation for Julia Finite Element Code, \texttt{FEM} Module}
\author{Jack Chessa}
%\date{} % Activate to display a given date or no date (if empty),
         % otherwise the current date is printed 

\begin{document}
\maketitle
\texttt{REALTYPE},  \texttt{IDTYPE}

\section{Basic Finite Element Routines}
\subsection{Shape Function Routines}
Routines that generate element shape functions and gradient with respect to the element coordiante system
\begin{itemize}
\item \texttt{shape\_}\textit{elemtype}\texttt{!(N, xi)}
\item \texttt{dshape\_}\textit{elemtype}\texttt{!(dNdxi, xi)}
\end{itemize}
Element types (\textit{elemtype}) are \texttt{line2}, \texttt{line3}, \texttt{tria3}, \texttt{tria6}, \texttt{quad4}, \texttt{tetra4},   \texttt{tetra10}, \texttt{hexa8}, \texttt{hexa20}.

\subsection{Quadrature functions}
\begin{itemize}
\item \texttt{qrule1d( npts )}
\item \texttt{qrulecomp(sdim,npt1,npt2=npt1,npt3=npt1)}
\item \texttt{qruletria(quadorder)}
\item \texttt{qruletetra(ord)}
\end{itemize}

\subsection{Finite element operations}
\begin{itemize}
\item \texttt{setsctr!(sctr, conn, nn, ndofpn)}
\item \texttt{isojac!(jmat, dNdxi, coord, nn, sdim=3, edim=sdim)}
\item \texttt{gradbasis!(dNdx, dNdxi, coord, nn, sdim=3, edim=sdim)}
\item \texttt{formbmat!(B, dNdx, nn, sdim=3)}
\item \texttt{formnv!(Nv, N, sdim=size(Nv,2), nn=length(N))}
\item \texttt{fesolve!(K, f, ifix)}
\item \texttt{fesolve!{T}(K, f::Array{T,1}, ifix::Array{Int,1}, ival::Array{T,1})}
\end{itemize}

Delayed assembly matrix
\begin{verbatim}
mutable struct DelayedAssmMat
   Kg::Array{REALTYPE}
   Ig::Array{IDTYPE}
   Jg::Array{IDTYPE}
   ii::Array{IDTYPE,2}
   jj::Array{IDTYPE,2}
   kk::Array{IDTYPE,1}
end
\end{verbatim}
\texttt{DelayedAssmMat} functions
\begin{itemize}
\item \texttt{addlocalmat!(K, ke, sctr)}
\item \texttt{addlocalmat!(K::DelayedAssmMat, ke, sctr)}
\item \texttt{getCSCMat(K::DelayedAssmMat)}
\end{itemize}

\subsection{Dof Maps}

\begin{verbatim}
struct FixedDOFMap <: AbstractDOFMap
  ndofpn::UInt8
  numdof
end
\end{verbatim}


\begin{verbatim}
struct VariDOFMap <: AbstractDOFMap
  map::Dict{Tuple{IDTYPE, UInt8}, IDTYPE}
  ndof::IDTYPE
end
\end{verbatim}

DOF Map functions
\begin{itemize}
\item \texttt{FixedDOFMap(n)}
\item \texttt{gdof(map::FixedDOFMap, nid, ldof)}
\item \texttt{sctrvct!(sctr, map::FixedDOFMap, conn, ldofs)}
\end{itemize}

\section{Material Routines}
\begin{itemize}
\item \texttt{cmatpstress!(C, E, nu)}
\item \texttt{cmatpstrain!(C, E, nu)}
\item \texttt{cmat3d!(C, E, nu)}
\end{itemize}

\subsection{Material Law Structures and Functions}

\begin{verbatim}
struct MatAbaqus <: AbstractMaterial
  rho::Union{Array{REALTYPE,2}, REALTYPE}
  elastic::Dict{String, Union{Array{REALTYPE,2}, REALTYPE}}
  plastic::Array{REALTYPE,2}
  conductivity::Union{Array{REALTYPE,2}, REALTYPE}
  specificheat::Union{Array{REALTYPE,2}, REALTYPE}
end  
\end{verbatim}

\begin{verbatim}
struct MatElastic <: AbstractMaterial
  rho::REALTYPE
  E::REALTYPE
  nu::REALTYPE
  G::REALTYPE
  ge::REALTYPE
  alpha::REALTYPE
  tref::REALTYPE
end
\end{verbatim}

\begin{verbatim}
struct MatThermal <: AbstractMaterial
  rho::REALTYPE
  cp::REALTYPE
  kappa::REALTYPE
end
\end{verbatim}

\begin{verbatim}
struct MatThermoElastic  <: AbstractMaterial
  rho::REALTYPE
  E::REALTYPE
  nu::REALTYPE
  G::REALTYPE
  alpha::REALTYPE
  tref::REALTYPE
  cp::REALTYPE
  kappa::REALTYPE
end
\end{verbatim}

\subsection{Element Property  Structures and Functions}
\begin{verbatim}
struct SectionTruss <: AbstractSection
  mat::AbstractMaterial
  A::REALTYPE
  J::REALTYPE
  c::REALTYPE
end
\end{verbatim}

\texttt{SectionTruss(mat, A; J=0., c=0.) = SectionTruss(mat, A, J, c)}

\begin{verbatim}
struct SectionSolid <: AbstractSection
  mat::AbstractMaterial
end
\end{verbatim}

\section{Element Type Specific Routines}

Basic structure of element type (ELEM\_TYPE)
\begin{verbatim}
struct ELEM_TYPE <: AbstractElement
	conn::Array{IDTYPE,2}
	sect::SECTION_TYPE
	eid::Dict{IDTYPE,IDTYPE}
	QP::Array{REALTYPE,1}
	QW::Array{REALTYPE,1}
	MP::Array{Dict,1}
end
\end{verbatim}

\begin{itemize}
\item \texttt{numelem(elem)}
\item \texttt{numelem(elem::AbstractElement) }
\item \texttt{numelem(elem::Array{Int,2}) }
\item \texttt{elemorder(elem::AbstractElement) }
\item \texttt{elemqrule(e::AbstractElement, ord=2)} 
\item \texttt{elemnne(e::AbstractElement) }
\item \texttt{elemnqpt(e::AbstractElement) }
\item \texttt{elemedim(e::AbstractElement) }
\item \texttt{elemsdim(e::AbstractElement) }
\item \texttt{elemvdim(e::AbstractElement)}
\item \texttt{elemndofpn(e::AbstractElement)}
\item \texttt{elemldofs(e::AbstractElement) }
\item \texttt{elemndofpe(e::AbstractElement) }
\item \texttt{elemcmat(e::AbstractElement)}
\item \texttt{elembasis!(e::AbstractElement, N, q) }
\item \texttt{elemgradbasis!(e::AbstractElement, N, q) }
\item \texttt{elemshape(e::AbstractElement)} : Returns shape functions at all quadrature points
\item \texttt{elemdshape(e::AbstractElement)}: Returns gradient shape functions, w.r.t. the element basis, at all quadrature points
\item \texttt{addstiffness!(elem::AbstractElement, node::Array{REALTYPE,2}, K, dofmap=FixedDOFMap(6))}
\end{itemize}

Element types (\texttt{<:AbstractElement})
\begin{itemize}
\item \texttt{Truss3D2P}
\item \texttt{FQuad4P }
\item \texttt{PEQuad4P}
\item \texttt{CHexa8P}
\end{itemize}

\section{Boundary Condition Structures and Functions}
\begin{verbatim}
struct SPC <: AbstractBC
  nid::IDTYPE
  dofs::Array{UInt8,1}
  val::REALTYPE
end  
SPC(nid, dofs) = SPC(nid, dofs, 0.0)
SPC(nid) = SPC(nid, Array{Uint8,1}(1:6), 0.0)
\end{verbatim}

\begin{verbatim}
struct SPC2 <: AbstractBC
	nids::Array{IDTYPE,1}
	dofs::Array{UInt8,1}
	val::REALTYPE
end  
\end{verbatim}

\begin{itemize}
\item \texttt{function add\_spcs!(spcs::Array{SPC,1}, nids, dofs, val=0.0)}
\item \texttt{function gen\_spcs(nids, dofs, val=0.0)}
\item \texttt{function getifix(spcs::Array{SPC,1}, dofmap=FixedDOFMap(6))}
\item \texttt{getifix(s::SPC2, dofmap=FixedDOFMap(6))}
\item \texttt{getifix(spcs::Array{SPC2,1}, dofmap=FixedDOFMap(6))}
\end{itemize}

\begin{verbatim}
struct LoadPt <: AbstractBC
  nid::IDTYPE
  force::Array{REALTYPE,1}
  ldof::Array{UInt8,1}
end 
LoadPt(nid, f) = LoadPt(nid, f, 1:length(f))
\end{verbatim}

\begin{verbatim}
struct LoadDist <: AbstractBC
  face::AbstractElement
  trac::Array{REALTYPE,1}
end
\end{verbatim}

\begin{itemize}
\item \texttt{addrhs!(loads::Array{LoadPt}, f, dofmap=FixedDOFMap(6))}
\end{itemize}

\end{document}
